\documentclass[a4j]{jsarticle}

\usepackage{listings}
\usepackage{color}
\usepackage{ascmac, here, txfonts, txfonts}
\usepackage[dvipdfmx]{graphicx}
\usepackage{comment}
\usepackage{otf}

\lstset{
  basicstyle={\ttfamily},
  identifierstyle={\small},
  commentstyle={\smallitshape},
  keywordstyle={\small\bfseries},
  ndkeywordstyle={\small},
  stringstyle={\small\ttfamily},
  frame={tb},
  breaklines=true,
  columns=[l]{fullflexible},
  numbers=left,
  xrightmargin=0zw,
  xleftmargin=3zw,
  numberstyle={\scriptsize},
  stepnumber=1,
  numbersep=1zw,
  lineskip=-0.5ex
}

\renewcommand{\lstlistingname}{ソースコード}


\begin{document}

\begin{titlepage}
  \title{情報特別演習\ajRoman{2} 最終レポート\\ Kinectによる自然な姿勢推定の実現}
  \author{情報科学類2年\\堤海斗}
  \maketitle
  \thispagestyle{empty}
  
\end{titlepage}

\begin{comment}
  ・演習概要
    ・演習テーマと背景
    ・演習目的
  ・演習手法
    ・FKによる姿勢推定
    ・自然な姿勢推定
      ・IKの導入
      ・キャリブレーションの導入
      ・フィルターの導入
  ・演習結果とまとめ
    ・実装結果
    ・現在取り組んでいる項目
    ・今後の展望
    ・演習で得られたこと
\end{comment}

\section{演習概要}

\subsection{演習目的}

今年の情報特別演習\ajRoman{2}では、「Kinectによる自然な姿勢推定の実現」
というテーマで1年間演習をしてきた。
本演習の目的は、Kinect v2というデバイスを用いて人の動きをセンサリングし、
そのデータを人型アバターに反映するという一連のシステム、つまり
モーションキャプチャシステムの実装において、
より自然な姿勢推定を実現することを目的としている。

本演習では以下の環境を用いる

\begin{itemize}
  \item Kinect v2
  \item Unity 2018.4.xx 
\end{itemize}

\subsection{演習の背景}


\section{演習手法}

\subsection{FKによる姿勢推定}

\subsection{自然な姿勢推定}

\subsubsection{IKの導入}

\subsubsection{キャリブレーションの導入}

\subsubsection{フィルターの導入}

\section{演習結果とまとめ}

\subsection{実装結果}

\subsection{現在取り組んでいる項目}

\subsection{今後の展望}

\subsection{本演習で学んだこと}


\end{document}