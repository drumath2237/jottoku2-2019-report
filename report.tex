\documentclass[a4j]{jsarticle}

\usepackage{listings}
\usepackage{color}
\usepackage{ascmac, here, txfonts, txfonts}
\usepackage[dvipdfmx]{graphicx}
\usepackage{comment}
\usepackage{otf}

\lstset{
  basicstyle={\ttfamily},
  identifierstyle={\small},
  commentstyle={\smallitshape},
  keywordstyle={\small\bfseries},
  ndkeywordstyle={\small},
  stringstyle={\small\ttfamily},
  frame={tb},
  breaklines=true,
  columns=[l]{fullflexible},
  numbers=left,
  xrightmargin=0zw,
  xleftmargin=3zw,
  numberstyle={\scriptsize},
  stepnumber=1,
  numbersep=1zw,
  lineskip=-0.5ex
}

\renewcommand{\lstlistingname}{ソースコード}


\begin{document}

\begin{titlepage}
  \title{情報特別演習\ajRoman{2} 最終レポート\\ Kinectによる自然な姿勢推定の実現}
  \author{情報科学類2年\\堤海斗}
  \maketitle
  \thispagestyle{empty}
  
\end{titlepage}

\begin{comment}
  ・演習概要
    ・演習テーマと背景
    ・演習目的
  ・演習手法
    ・FKによる姿勢推定
    ・自然な姿勢推定
      ・IKの導入
      ・キャリブレーションの導入
      ・フィルターの導入
  ・演習結果とまとめ
    ・実装結果
    ・現在取り組んでいる項目
    ・今後の展望
    ・演習で得られたこと
\end{comment}

\section{演習概要}

\subsection{演習目的}

今年の情報特別演習\ajRoman{2}では、「Kinectによる自然な姿勢推定の実現」
というテーマで1年間演習をしてきた。
本演習の目的は、Kinect v2というデバイスを用いて人の動きをセンサリングし、
そのデータを人型アバターに反映するという一連のシステム、つまり
モーションキャプチャシステムの実装において、
より自然な姿勢推定を実現することを目的としている。

本演習では以下の環境を用いる

\begin{itemize}
  \item Kinect v2
  \item Unity 2018.4.xx 
\end{itemize}

\subsection{演習の背景}

前述のとおり本演習ではKinect v2というデバイスを使うのだが、
このデバイスはMicrosoftが開発していた赤外線カメラデバイスの1種である。
このデバイスではカラーカメラによって通常のカラー画像をはじめ、
赤外線による深度画像が取得可能である。
加えて深度画像を解析して人間を検知し、人の関節データを取得することが可能である。
Kinectだけでも姿勢推定は可能なのだが、
今回はUnityを用いてVRMやMMDといった人型アバターモデルに適用することで、
モーションキャプチャをしようと考えた。

このようなテーマにした理由として、昨今のVTuberブームが挙げられる。
2017年下旬から”VTuber”とよばれるコンテンツが話題になっていた。
VTuberとは人間の動きをモーションキャプチャによってバーチャルなキャラクターに反映し、
Youtubeなどの動画配信プラットフォームで動画投稿や配信活動をするコンテンツである。

私はこのVTuberにとても興味があり、
このようなものを自分でも実現してみたいと思い、このテーマを選択した。
Kinect v2は最初に今回演習で担当をして下さっている志築先生のIPLABからお借りして、
その後に、夏の長期休み明けからは自分の私物を使った。

\subsection{モーションキャプチャについて}

本演習ではモーションキャプチャシステムを作ることが目的であるので、
そもそもモーションキャプチャとは何かについて説明する。

前節で少し言及した通り、モーションキャプチャとは
人をはじめとしたものの動き(モーション)をセンサリングで取得する(キャプチャ)ことである。
モーションキャプチャを利用して、前述したVTuberやゲームキャラクターのモーション作成、
ジェスチャーインプットを使ったアプリケーションの作成などができる。

モーションキャプチャを行うには、モーションキャプチャができるデバイスが必要だが、
どの程度のモーションキャプチャが必要かどうかによって必要な機材が変わってくる。
例えば、上半身だけのモーショントラッキングであれば、
顔の位置と表情をトラキングできればよいのでスマホ1台でもできる。
そこに手の動きを加えたければLeapMotionなどのデバイスを使う。
また、立った姿でトラッキングをするとなると、VRセットが活用できる。
Oculus Questなどのデバイスを使うことによって
VRアプリケーションを操作するために最適なモーションキャプチャが可能である。

今回の演習では、全身を使った、いわゆるフルトラッキング・モーションキャプチャをする。
フルトラッキングができるデバイスの例を以下に示す。

\begin{itemize}
  \item VICON
  \item OptiTrack
  \item Vive Tracker
  \item Perception Nueron
  \item RealSense
  \item iPhone 11 Pro
\end{itemize}

精度の差はあれど、これらのデバイスをセンサーとして用いることで
モーションデータの取得が可能であり、
取得したデータを利用してアプリケーションを開発する。

本演習ではKinect v2を使ってモーションキャプチャをするわけだが、
前述のとおり、Kinect v2では、Kinect自身でモーションキャプチャが完結しており、
モーションデータをUnityやC++アプリケーションでリアルタイムに取得可能である。
Kinect v2の主な特徴は以下のとおりである。

\begin{itemize}
  \item 非接触型センサーである
  \item Unity用のSDKが配布されている
  \item 複数人のモーションキャプチャが同時にできる
  \item 開発が終了している
\end{itemize}

非接触型センサーとは、対象物にデバイスを接触させないでセンサリングする方法である。
通常、モーションキャプチャデバイスでは、トラッカーと呼ばれるデバイスを
関節に装着したりマーカーを全身に着けたりするのだが、
Kinectでは深度情報から姿勢推定をするため、そのような手間がない。
したがって他のデバイスと比べてユーザ・エクスペリエンスが高いという特徴がある。

また、Kinectは開発元のMicrosoftからUnity用のSDKが配布されており、
Unityのプロジェクトにインポートすることで容易に開発を始めることができる。
UnityではScene内にプレハブ展開し、ゲームオブジェクトにスクリプトをアタッチすることで
UnityのAPIを利用したプログラムを実行できるのだが、
ポーリング式でセンサーデータを取得する場合には
プレハブもスクリプトも用意してある。

しかし、Kinect v2はサポートが終了しており、
開発が進められていないというデメリットがある。
実際、Kinectの用途として、深度情報から点群情報を解析して利用することが多く、
BodyTrackingに関しては文献が少ないというデメリットもある。

\section{演習手法}

\subsection{FKによる姿勢推定}

モーションキャプチャを行うにあたって、まず初めにFKという手法を用いて
アバターのボーンを制御した。
FKは"Forward Kinematics"の略で、日本語訳すると"順運動学"ということになる。
順運動学とはボーン制御手法の1種で、
目的の姿勢を親のボーンデータから推定していく手法を指す。

例えば、人間の腕のような2ボーンFKを考えると、
まず肩のボーンの回転を取得する。そして肩のボーンを親に持つ肘のボーン
の位置が決まる。
そして肘の回転から手首の位置が決まる、という風にボーンを推定・制御していく。

FKはとても直観的なロジックなため実装自体は簡単だが、
センサリングを行う上で生じる誤差が末端のボーンに行くにつれ蓄積されていくという
致命的な欠点がある。
また、FKでは推定するすべての関節データが必要になってくるため、
広く使われているようなトラッカーを使うモーションキャプチャには適用するのが難しい。

実際に、Kinectのモーションデータ(回転データ)をそのまま適用してみた結果、
以下の画像のようになってしまった。

\includegraphics[width=10cm]{img/fk-noise}

これは肩のボーンの回転が適当に取得できなかったため、
それが肘、手首のデータの誤差と共に蓄積されてしまった結果である。
本来ならば気を付けの姿勢をしているが、肩が回転しなかったため、
肘が横に行ってしまっている。


\subsection{自然な姿勢推定}

前節のように、FKだけだとうまくボーンデータを反映することができなかった。
ここでは本演習の主題でもある「自然な姿勢推定」を実現するために導入した手法を説明していく。

\subsubsection{IKの導入}

FKでは、一番親のボーンから末端のボーンまでを推定していたが、
IKはその逆で、末端のボーンからその他のボーンデータを推定していく手法で、
"Inverse Kinematics"の略である。逆運動学とも言う。

IKを使うことにより、FKにおける問題を解消することができる。
FKでは末端のボーンに行くにつれ誤差が蓄積するという問題があったが、
IKは末端のボーンのデータによって他のボーンデータを推定するため、
誤差はボーンデータ1つ分で済む。
また、末端のボーンデータをとるためのトラッカーがあればよいので、
機材の観点からも利用の幅が広がるのである。

しかしFKよりもIKの実装には3次元の幾何学的な処理が必要になるため
実装は難易度が高くなり、計算量も多くなる。
そこで今回はUnityのMecanimというシステムを利用した。
Mecanimは、Unityで標準的に実装されているアバター制御のためのAPIで、
モデルがHumanoidとしてリギングされており、適切なAnimationControllerが設定されていれば、
IKに必要な末端のボーンの位置/回転、適用する重さを設定することにより
容易にIK制御を実現できる。


\subsubsection{キャリブレーションの導入}

IKを導入すると、必然的に回転の他に位置の概念が重要になってくる。
モーションキャプチャのターゲットとなる人間と
モーションデータを反映するアバターとの位置合わせを行う必要がある。
そこでキャリブレーションとレジストレーションという二つの処理を導入する。

キャリブレーションとは本格的なセンサリングをする前に、
センサーの標準状態を測定することを言う。
モーションキャプチャでのキャリブレーションは、
ターゲットとなる人間のボーンデータをあらかじめ測定し、
体格を把握することを指す。

モーションキャプチャをするときにモーションデータを反映する際に
体格の違いによって、例えば腕の長さが違うことにより
ターゲットの人間は腕を伸ばしているのに、
アバターは腕が伸び切っていないような状態になることがある。
それを解決するためにまず人間の体形を把握し、
キャリブレーションした体形から、相対的にアバターの関節の位置を割り出す必要がある。
相対的なボーンデータを割り出し、アバターに反映する操作をレジストレーションと呼ぶ。

\subsubsection{フィルターの導入}

IKとキャリブレーションに加えて、センサリングによって生じたノイズを除去するために
フィルターという手法を導入した。
フィルターとは、ノイズの入ったデータから真に得たい情報を抽出するための手法で、
今回は指数平滑化フィルターというフィルターを導入した。
指数平滑化フィルターは以下の式で与えられるフィルターで、
IRフィルターの一種である。

\begin{eqnarray*}
EMA[x_{n+1}] &=& y_{n+1} \\
             &=& \alpha x_{n+1} + (1-\alpha)y_{n}
\end{eqnarray*}

\begin{eqnarray*}
\tilde{x} &=& DEMA[x_{n+1}] \\
          &=& 2DEMA[x_{n+1}]-EMA[EMA[x_{n+1}]]
\end{eqnarray*}

この$\tilde{x}$で与えられている式が、指数平滑化フィルターの定義である。
このフィルターによってノイズの乗った生のセンサーデータを滑らかにすることができる。

\section{演習結果とまとめ}

\subsection{実装結果}

本演習では、前章で述べたような手法を導入したことでモーションキャプチャを実装することができた。
最終的に実装したモーションキャプチャシステムは以下のような機能を備えている。

\begin{itemize}
  \item Kinect v2からリアルタイムでモーションを反映できる
  \item モーションデータをVRMやMMDといったアバターフォーマットに反映できる
  \begin{itemize}
  \item 体形の違うモデルにも適用可能
  \item モーションデータは滑らかに補完されている
  \end{itemize}
\end{itemize}

なお、導入した指数平滑化フィルターでの係数$\alpha$はUnityでの10フレーム分の時間を割り当てている。
この係数$\alpha$は大きくしすぎてしまうとが発生してしまうため、経験則からなるべく小さくするように設定した。

以下の画像は、体形の違う複数のモデルでの適用例である。

\includegraphics[width=10cm]{img/kekka.png}

左から順に、VRMの公式モデル"千駄ヶ谷篠"、自作のVRMモデル、体形の違う二つのMMDモデル、となっている。
見てみると、4つのモデルは大体同じような動きをしていることがわかるが、
右二つのMMDモデルに関しては、腕の長さが短くとれてしまっていることがわかる。

これは右2つのMMDのボーン構造と左2つのVRMのボーン構造の違いによって引き起こされたものだという考察ができる。
右2つのMMDモデルを調べてみると、胸のボーンが高い位置にあることが確認された。
そのため、胸を基準に腕の相対的なボーンを推定すると、腕が短くとれてしまった事が原因だと考えられる。

\subsection{現在取り組んでいる項目と今後の展望}

現在は以下の3つの項目に取り組んでいる。

\begin{itemize}
  \item VRMとMMDモデルのボーン構造を統一的に処理でいるキャリブレーションロジック
  \item 保守のしやすい設計への落とし込み
\end{itemize}

前述のとおり、最終的に実装したモーションキャプチャでは
VRMとMMDの差により腕の長さに違いが出てしまった。
これを解決するために、今回採用した「胸から手首へ」のキャリブレーションから
「肩から手首へ」のキャリブレーションに移行することで、
胸のボーンに関係なくキャリブレーションが行える。

2つ目の項目については前から取り組んでいたのだが、
センサーの入力、キャリブレーション、モデルへの仲介、データの加工、IK制御といったようにな
順番のシステムに対して、一部の機能を変更しても全体に影響がないような賢い設計を考えており、
そちらへの移行作業を進めている。

また、今後の展望として、以下のような項目を考えている。

\begin{itemize}
  \item より自然なボーン制御
  \begin{itemize}
    \item 腕を上げる時に肩も一緒にあげる処理
    \item 地面と足との設置感の再現
  \end{itemize}
  \item 手、顔、表情のトラッキング
  \item カルマンフィルタなどの高度な統計処理の導入
\end{itemize}

\subsection{最終発表会で受けた質問}

最終発表会では以下のような質問を受けた。

\begin{itemize}
  \item ボーンというのは関節という意味なのか?
  \item 最終的に実装したいのは人間の動きの計測なのか?
  \item リアルタイムで動作するのか?
  \item フィルターの係数$\alpha$はどんな値なのか?
\end{itemize}

1つ目の質問は内容としてはあっており、ボーンというのは関節データの回転と位置を取得したデータの事であるという趣旨の説明をした。

2つ目の質問に対しては、「人間の動きをキャプチャしアバターに適用する、モーションキャプチャの一連の処理を行うシステムの開発を目的としている」という説明をした。

3つめは、リアルタイムに動作する趣旨の説明をしたが、フィルターによって多少のレイテンシが生じることも説明した。

4つめの質問に対しては、前節で説明した通り、Unityでの10フレーム分の時間を代入していることを説明した。

\subsection{本演習で学んだこと}


\end{document}